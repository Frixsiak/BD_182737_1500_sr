%\documentclass[a4paper, 10pt]{article}
%\usepackage[MeX]{polski}
%\usepackage[utf8]{inputenc}
%\begin{document}
%\ldots Słynne równanie Einsteina
%\begin {equation}
%e = m \cdot c^2 \; ,
%\end{equation}
%\end{document}


%\documentclass[a4paper, 10pt]{article}
%\usepackage[MeX]{polski}
%\usepackage[utf8]{inputenc}
%\begin{document}
%\begin{equation}
%\ldots którego wynikiem jest prądowe prawo Kirchhoffa:
%\sum_{k=1}{n} I_K = 0\; .
%\end{equation}
%\end{document}



%\documentclass[a4paper, 10pt]{article}
%\usepackage[MeX]{polski}
%\usepackage[utf8]{inputenc}
%\begin{document}
%\begin{equation}
%\ldots co ma określone zalety.
%I_D = I_F - I_R
%\end{equation}
%jest rdzeniem innego modelu tranzystora. \ldots
%\end{document}


%\documentclass[a4paper, 10pt]{article}   
%\usepackage[MeX]{polski}                 
%\usepackage[utf8]{inputenc}              
%\begin{document}
%Nie\-bie\-sko\-bia\-ło\-zie\-lo\-%
%no\-nie\-bie\-ski
%\end{document}                     


%\documentclass[a4paper, 10pt]{article}   
%\usepackage[MeX]{polski}                 
%\usepackage[utf8]{inputenc}              
%\begin{document}                         
%Numer mojego telefonu zmieni się na \\\mbox{0116 291 2319}
%\\Parametr \mbox{\emph{nazwa}} to nazwa pliku
%\\To jest w {\fbox {ramce}} 
%\end{document}


%\documentclass[a4paper, 10pt]{article}   
%\usepackage[MeX]{polski}                 
%\usepackage[utf8]{inputenc}              
%\begin{document}                         
%\today
%\TeX
%\LaTeX
%\LaTeXe
%\end{document}



\documentclass[a4paper, 10pt]{article}   
\usepackage[MeX]{polski}                 
\usepackage[utf8]{inputenc} 
\usepackage{textcomp}  
\usepackage[gen]{eurosym}           
\begin{document}      
"Please press the 'X' key"
\newline,,Przechodź tylko po <<zebrach>>"
\\niebiesko {\dywiz} czarny
\\http://www.rich.edu/\~{}bush \\
\\http://www.cleaver.edu/$\sim$demo
\\Jest $-30\,^{\circ}\mathrm{C}$.
Niedługo zacznie mrozić.
\\ \texteuro
\\Nie tak ..., ale raczej tak:\\ Nowy York, Tokio, Budapeszt, \ldots
\\ jak lepiej: geografii czy geograf\mbox{}ii?
\\ H\^otel, na\"\i ve, \'el\'eve \\ Sm\o rrebr\o d, !'Se\~norita!, \\ Sch\"onbrunner Schlo\ss{}
Stra\ss e \\
\k{a} \'c \k{e} \l{} \'n \'o \'s \'z \.z \k{A} \'C \k{E} \L{} \'N \'O \'S \'Z \.Z 
\\ Pan Kowalski ucieszył się\\ na jej widok (zob.~Rys. ~5).\\ Podoba mi się JAVA\@. A~tobie?


\end{document}     
              
